\section*{Introduction}
% Dire cest quoi leffet cheerios etc \ldots
    %What are cheerios? Is the cheerios effect really an effect or something we created in our reality to rationalise the sogyness of our cereals. Will we be able to eat cereals without making them soggy since we will be so foccused at the cheerios effect. Is the cheerios effect a conspiracy fro cheerios to make us eat soggy cereals or mayb ethey are thirsty and want more time in liquid or maybe they miss their friends inside the bag so the y curve the water to reach them but there was something that they didnt know; the fact that researchers would modelise it and one day two students would choose this effect do simulate it. Maybe that was the objective overall trying to put sense in our worthless lifes. Why would the cheerios be intersted in our lifes? Are we interested in our lifes? Who knows \ldots  
    %Maybe one day we will see what this project meant to but for now we are sailing sailing on the unexpected and in a world filled with possibilities only exploring some possibilities leaving infinite ones untouched. Maybe one day all that will make sense but for now the only thong that makes sense is this paper or maybe not even this so would that mean that our life doesnt have sense/meaning? That is left to the reader maybe it does maybe it doesn't. Maybe it depends how you look at it. It is all filled with maybes however you might look at it as someone scottis and i quote "No you you seee i'm talking facts here i dont do if buts and maybes i do absolutes..." and in that case i have only one thing to say in this paper that should be the case. And with all that said i am leaving you with this paper and a little bit of existentialism, nihilism and our besst wishes about the future.
    Dans le cadre de l'UE Projet en Calcul Scientifique Numérique, nous devions travailler sur un projet, afin de nous apprendre plus en détail, la programmation et le calcul numérique avec un langage compilé, le C. Notre sujet était sur l'"Effet Cheerios", ou l'interaction d'objets à la surface d'un liquide par l'effet de la gravité et la déformation interfaciale. Cet effet se caractérise par la tension d'une surface liquide sous le poids d'un objet, par exemple une punaise sur l'eau. Lorsque nous ajoutons plusieurs objets sur la même surface, à distance plus ou moins grande, les objets vont potentiellement s'attirer puis créer des tas mobiles. Ce phénomène est notamment visible avec des céréales dans du lait, d'où le nom de Cheerios, célèbre marque de céréales américaine. Pour réaliser à bien ce projet nous avons du faire de nombreuses recherches sur la mécanique des fluides, les collisions inélastiques et nous avons également du faire un travail conséquent sur l'optimisation de notre algorithme. Nous allons vous raconter comment fonctionne l'effet Cheerios et comment nous avons calculé les forces d’interaction. Puis nous vous expliquerons les méthodes numériques principales utilisées dans notre algorithme. Enfin nous vous expliquerons notre algorithme en pointant ses défauts, et nous finirons par une analyse critique des résultats obtenus.