\documentclass[a4paper, 11pt, oneside]{article} % A4 paper size, default 11pt font size and oneside for equal margins

\usepackage[utf8]{inputenc} % Required for inputting international characters
\usepackage[T1]{fontenc} % Output font encoding for international characters
\usepackage[french]{babel}

\usepackage{hyperref} %For hyperlinks 
\hypersetup{pdfborder=0 0 0}

\usepackage{graphicx}  % Pour images
\graphicspath{{Figures/}}

\newcommand{\comment}[1]{}%pour faire des comentaires a plusueures lignes 

\begin{document} 

\newpage
\section{Introduction}
    Dire cest quoi leffet cheerios etc...

\section{nescescites}
    \subsection{Effet Cheerios}
    On mets les formules et peut etre demontrer ou ils viennes et sourtout les cas ou on peux utiliser ces formulles les cas ou ca marche pas etc...
    \subsection{integration de verlet}
    On prouve lintegration de verlet et montre que on peux lutiliser pour notre probleme
    \subsection{Collsisions}
    Expliquer comment on a deduit que les collisions etait des collisions inelastic parfait et metre les equations utilise

\section{Comment on a concue notre probleme}
    \begin{itemize}
        \item On a pris l'interaction des forces totale sur chaque particule par la fonction dans l'article "Cheerios effect"
        \item et de ca on deduis la force que reagis a chaque cheerios pour un pas de temps 
        \item Check si il ya des collisions ou pas et si il ya on change les proprietes des cheerios par rapport aux collisions
        \item De la force en utilisant l'integration de verlet et le principe fondamentale de la dynamique somme forces = derive(masse*vitesse) on peux changer les positions des cheerios
    \end{itemize}
\section{Conclusion}
\end{document}
